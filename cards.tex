% © 2022 TNG Technology Consulting
%
% SPDX-License-Identifier: CC-BY-4.0
% 
% Access & Secrets
\AccessSecrets{2} {We grant permissions to 3rd parties (e.g. CI/CD systems), but do not review them regularly.}
\AccessSecrets{3} {Our secrets are long-lived and can be reused when they get leaked.}
\AccessSecrets{4} {We don't enforce strong passwords for cloud access, so brute-forcing is possible.}
\AccessSecrets{5} {We (as developers) have access to technical credentials.}
\AccessSecrets{6} {We don't propagate changes in permissions quickly enough throughout the whole system.}
\AccessSecrets{7} {We can't trace back whether authenticated users/developers granted themselves additional permissions.}
\AccessSecrets{8} {We don't restrict permissions (developers, technical users) to the minimum, allowing for a privilege escalation.}
\AccessSecrets{9} {Our Identity and Access Management lets authenticated users/developers grant themselves additional permissions.}
\AccessSecrets{10}{We don't enforce MFA for developer access.}
\AccessSecrets{J} {Our deployment artifacts contain secrets that can be extracted.}
\AccessSecrets{Q} {Our Identity and Access Management is too complex.}
\AccessSecrets{K} {We don't use an established solution for credential management.}
\AccessSecrets{A} {Our source code contains secrets.}
%
% Delivery
\Delivery{2} {We don't know the versions of our dependencies or whether they are up to date.}
\Delivery{3} {We include unneeded dependencies when deploying our system (test, build, compile-time dependencies).}
\Delivery{4} {We don't know the source repository of our dependencies (dependency confusion).}
\Delivery{5} {We don't know how a new version of a dependency changes our system (rogue dependencies).}
\Delivery{6} {Our system can be re-deployed by a change in an external dependency.}
\Delivery{7} {We don't know whether our dependencies introduce security issues (missing vulnerability scans).}
\Delivery{8} {We use outdated dependencies of our runtime platform (OS, container image, serverless runtime).}
\Delivery{9} {We use untrustworthy dependencies (unmaintained, used by too few people, developed by single developers, ...).}
\Delivery{10}{We don't limit ingress or egress when running CI pipelines.}
\Delivery{J} {We don't know when someone injects code into our codebase.}
\Delivery{Q} {We are not certain which code/artifacts we are deploying (source code integrity).}
\Delivery{K} {We won't notice when a deployment is started from a developer account.}
\Delivery{A} {We won't notice when someone alters the deploy pipeline.}
%
% Recovery
\Recovery{2} {-}
\Recovery{3} {We have backups but do not check regularly whether we can restore them or not.}
\Recovery{4} {We have no backups for our infrastructure (IaC and its state).}
\Recovery{5} {We have no backups of our application data.}
\Recovery{6} {We have no backups for our secrets.}
\Recovery{7} {We cannot restore our infrastructure to a previous state.}
\Recovery{8} {We cannot restore our application to a previous state.}
\Recovery{9} {We cannot restore our complete environment to a previous state.}
\Recovery{10}{We don't create backups before deleting important data.}
\Recovery{J} {All our backups can be destroyed at once, due to lack of redundancy.}
\Recovery{Q} {We can't tell whether our backup has been modified.}
\Recovery{K} {We can have the same person deleting resources and their backups.}
\Recovery{A} {We have no disaster recovery plan.}
%
% Monitoring
\Monitoring{2} {-}
\Monitoring{3} {-}
\Monitoring{4} {-}
\Monitoring{5} {-}
\Monitoring{6} {We can't easily identify useful information in logs.}
\Monitoring{7} {We won't get an alert if an end user generates huge cloud bills for us.}
\Monitoring{8} {We don't notice if an authenticated attacker/developer deactivates or manipulates our tools for traceability.}
\Monitoring{9} {We don't know if an authenticated attacker/developer accessed the production environment.}
\Monitoring{10}{We cannot react to problems in time because our monitoring has blind spots.}
\Monitoring{J} {We need too long to figure out what an alert means.}
\Monitoring{Q} {We do not know how to react when our monitoring sends alerts.}
\Monitoring{K} {We can't access our logs if the production environment goes down.}
\Monitoring{A} {We write secrets/personal data to our logs.}
%
% Resources
\Resources{2} {-}
\Resources{3} {-}
\Resources{4} {We can't get contacted by our cloud provider in case of emergency.}
\Resources{5} {We don't regularly check compliance with our policy for using/configuring cloud resources.}
\Resources{6} {We have not configured any rate limits for our services.}
\Resources{7} {We have no configured resource limits.}
\Resources{8} {We can deploy applications with excessive capabilities.}
\Resources{9} {Our whole system can be affected by a single rogue service.}
\Resources{10}{We don't control ingress traffic.}
\Resources{J} {We don't control egress traffic.}
\Resources{Q} {Our production and staging environments connected, either directly or indirectly (e.g. via CI/CD).}
\Resources{K} {Our cloud resources are publicly exposed without any need.}
\Resources{A} {We have no clear policy for using/configuring cloud resources.}
